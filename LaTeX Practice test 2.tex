\documentclass[12pt,a4paper]{article}
\usepackage{amsmath,graphicx,color,amssymb}
\parindent=0mm
\parskip=2mm
\begin{document}
\title{\LaTeX Practice Test 2}
\author{Ihsan Mahmood}
\maketitle

\par For a given $\alpha$ and $\beta$, we have that $\displaystyle \tan(\alpha \pm \beta) = \frac{\tan(\alpha) \pm\tan(\beta)}{1 \mp \tan(\alpha)\tan(\beta)}$.

\par 	If a 1:1 function $\gamma$ exists that maps elements from the rationals $\mathbb{Q}$ to the elements of the natural numbers $\mathbb{N}$, that is $\gamma : \mathbb{Q} \rightarrow \mathbb{N}$, then $\mathbb{Q}$ is a countable set.

\par If $f$ is a function of $x$ and $y$, which are themselves functions of where $u$ and $v$, then
\begin{equation} 
  \frac{\partial f}{\partial x}=\frac{\partial f}{\partial u}\frac{\partial u}     {\partial x} + \frac{\partial f}{\partial v}\frac{\partial u}{\partial x}.
\end{equation}
Note that under wide conditions $\displaystyle \frac{\partial ^2 f}{\partial x\partial  y}=\frac{\partial ^2 f}{\partial y\partial x}$ 

\par l'Hopital's Rule states that if $\displaystyle \lim{_{x \rightarrow \infty}} q(x)= \lim{_{x \rightarrow \infty}} p(x)=0$
\begin{equation} \lim_{x \rightarrow \infty} \frac{q(x)}{p(x)}=\lim_{x \rightarrow \infty} \frac{q'(x)}{p'{x}}
\end{equation}
if that limit exists.
\par{ \parskip=0mm \parindent=6mm The surface area of the revolution of a curve $y(x)$ between limits x=a and x=b is given by}
\begin{equation*}
S= \int^a _b 2\pi y \sqrt{1+ (\frac{dy}{dx})^2} dx.
\end{equation*}
\par Integrating the function $\displaystyle \frac{1}{a+be^mx}$ w.r.t $x$ gives 
\begin{equation} \int \frac{1}{a+be^{mx}}dx=\frac{1}{am}(mx- \ln(a+be^{mx}))+ C
\end{equation}
The Binomial Theorem states that, for $|x|$ $<$ 1,
\begin{equation} (1-x)^-1=1-x+x^2-x^3+ \cdots = \sum^\infty_{n=1}(-1)^nx^n.
\end{equation}
\end{document}